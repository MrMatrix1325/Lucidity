\documentclass[12pt]{article}

\usepackage[utf8]{inputenc}
\usepackage{fancyhdr}
\usepackage{geometry}
\usepackage[frenchb]{babel}
\usepackage{libertine}
\usepackage[pdftex]{graphicx}
\usepackage{hyperref}
\usepackage{slashbox}
\usepackage[T1]{fontenc}
\usepackage{multirow}
\usepackage{graphicx}
\usepackage{array,multirow,makecell}
\setcellgapes{1pt}
\makegapedcells
\geometry{a5paper}
\geometry{top=1.5cm, bottom=1.5cm, left=1cm, right=0.5cm}

\begin{document}

\pagestyle{fancy}
\lhead{The Chamallow}
\rfoot{\leftmark}
\renewcommand{\footrulewidth}{0.4pt}
\newcolumntype{C}{>{\centering}X}
\rhead{Dossier d'exploitation}

\tableofcontents

\newpage

\section{Histoire}

\quad

En 3042, les scientifiques du monde entier se sont unis pour pouvoir partager toutes les connaissances accumulées au cours des siècles. Ainsi ils ont pu faire beaucoup de progrès, surtout dans le domaine spatial. Cela a permis à l’espèce humaine de commencer à coloniser l’espace. D’abord des planètes viables près de la planète Terre, puis de plus en plus loin, jusqu’à aller dans d’autres galaxies. Maintenant, ils ont commencé à coloniser une nouvelle galaxie, très très lointaine. Dans cette galaxie, des humains se sont installés sur une planète qu’ils ont appelée Lucidios. Cette planète est complètement différente de la Terre. Là-bas, ils ont trouvé des matériaux qui n’existent sur aucune des planètes déjà colonisées, ils ont aussi découvert un liquide multicolore qu’ils ont décidé de nommer la Lucidité.\\

Assez intrigués par cette dernière, des scientifiques ont fait des recherches dessus et après certains tests. Ils pensent que l’on pourrait l’utiliser comme une ressource dans la vie quotidienne des lucidiociens. Par exemple, en permettant de se déplacer d’un endroit à un autre en utilisant des plateformes mais aussi des portails. Effectivement, les scientifiques espèrent utiliser la Lucidité comme une source d’énergie. Ainsi en la mettant dans un mécanisme, cela permettrait d’actionner ce dernier. \\

Pour finaliser les tests, des salles ont été créées dans lesquelles il y a des objets avec lesquels il est possible d’interagir grâce à la Lucidité. Ils sont donc à la recherche de volontaires dont la mission sera de traverser ces salles en transportant de la Lucidité en bouteille. Le problème c’est qu’ils auront donc un stock limité sur eux, ils devront donc apprendre à maîtriser cette perte de ressource car ils ne pourront remplir leur réserve qu’au niveau de source de Lucidité. S’ils arrivent à traverser toutes les salles qui contiennent  des tests de difficulté croissante, leur mission sera réussite et ils pourront l’utiliser sur toute la planète et peut-être même sur les planètes voisines. Si vous êtes volontaire, n’hésitez pas à vous rendre sur notre site internet “lucidity.fr” pour vous inscrire.

\newpage


\section{Eléments}

\begin{flushleft}

\underline{Porte :}

Une porte permet d’accéder à la pièce suivante.


\quad


\underline{Source de lucidité :}

Cet élément te permet de recharger ta lucidité à 100%.

\quad

\underline{Check-points:}

Permet de reprendre à ce point si tu meurs.

\quad

\underline{Plateforme automatique}

Ces plateformes bougent toute seules.

\quad

\underline{Plateforme guidable}

Tu peux déplacer ces plateformes à l’aide des clips en en lançant un sur le mur et l’autre sur la plateforme.

\quad

\underline{Clip}

Permet de diriger la plateforme guidable.

\quad

\underline{Bouton}

Effectue une action dès que tu appuies dessus.

\quad

\underline{Endpoint}

Ce point permet de finir le niveau.

\quad

\newpage

\underline{Point apparition}

Permet de faire apparaître des plateformes.

\quad

\underline{Portail}

Permet de passer d’un endroit à l’autre.

\quad

\underline{Pistolet à clip}

Lance des clips.

\quad

\underline{Pistoler à portail}

Lance des portails.

\end{flushleft}

\newpage

\section{Les commandes}

Les commandes ci-dessous sont les commandes définies par défaut. Elles sont modifiables au lancement du jeu.

\quad



\begin{tabular}{|c|p{7 cm}|}
\hline Commandes & Actions \\
\hline Touches Z, Q, S et D & Se déplacer \\
\hline Touche F & appuyer sur un bouton ou ouvrir une porte\\
\hline Molette & Changer d'armes\\
\hline Clic droit & Lancer un portail ou un clip en fonction de l'arme choisie\\
\hline Clic gauche & Lance un portail\\
\hline Espace & Permet de sauter\\
\hline Touche échap & Menu pause \\
\hline

\end{tabular}


\newpage

\section{Installation}


Pour l'installation du jeu, insérer le DVD dans la zone prévue à cet effet. Le processus d'installation se lance automatiquement. Si cela ne se fait pas, aller dans ordinateur puis exécuter le DVD.


Le guide d'installation s'ouvre. Tout est bien indiqué, suivre les étapes d'installation.

\section{Désinstallation}


Un programme de désinstallation est spécialement conçu pour la désinstallation du jeu. Pour cela se rendre dans le dossier où le jeu est installé, et exécuter uninstall.exe pour désinstaller le jeu.

\newpage

\section{Notes}

\end{document}